\documentclass[a4paper]{article}
\usepackage[utf8]{inputenc} 
\usepackage[danish]{babel} 

\usepackage{amsmath}
\usepackage{amsfonts}
\usepackage{amssymb}
\usepackage{ulem}
\usepackage{verbatim}
\usepackage{graphicx}
\usepackage{xcolor}
\usepackage{marvosym}

\newcommand*{\escape}[1]{\texttt{\textbackslash#1}}

%\usepackage[margin=.7in]{geometry}

\usepackage{listings}
\lstdefinestyle{customc}{
  belowcaptionskip=1\baselineskip,
  breaklines=true,
  frame=L,
  xleftmargin=\parindent,
  language=C,
  showstringspaces=false,
  basicstyle=\footnotesize\ttfamily,
  keywordstyle=\bfseries\color{green!40!black},
  commentstyle=\itshape\color{purple!40!black},
  identifierstyle=\color{blue},
  stringstyle=\color{orange},
}

\lstdefinestyle{customasm}{
  belowcaptionskip=1\baselineskip,
  frame=L,
  xleftmargin=\parindent,
  language=[x86masm]Assembler,
  basicstyle=\footnotesize\ttfamily,
  commentstyle=\itshape\color{purple!40!black},
}

\lstset{escapechar=@,style=customc}
	
\title{PoP \\ \Large uge øvelser}
\author{Beate B. S.}
\begin{document}
\maketitle
\tableofcontents
\newpage
\section{ø.0}
\begin{lstlisting}
3.14 + 2.78;;
> 5.92
\end{lstlisting}

\section{ø.1}
\begin{lstlisting}
let a = 3.14 + 2.78;;
printfn "%A" a
>5.92
\end{lstlisting}
fsharpi og fsharpc virker, men kan ikke få mono til at virke endnu.
Ja ja, fsharpi er den hurtigste umiddelbart, men hvis man skal køre kode a' flere omgange vil det hurtigeste, i længden, være at compile lortet og så køre mono på den. 

\section{ø.2}
\begin{lstlisting}
aString = 'b', {'a'}; 
\end{lstlisting}
Følgende er gyldige resultater: 'b', 'ba', 'baa'.

\section{ø.3}
\begin{lstlisting}
number = ?any decimal number?;
operator = '+' | '-' | '*' | '/';
expression = number | number, operator, expression:
\end{lstlisting}
Der kan være vilkårlige mange gyldig expression med flere operator, så længe en operator er omgiver af tal. E.g. $3 * 6 - 7$ etc.

\section{ø.4}
\begin{lstlisting}
let hello = "hello";
let world = "world!";

printfn "%s %s" hello world;
> hello world!
\end{lstlisting}

\section{ø.5}
\begin{lstlisting}
3 + 1.0;;
\end{lstlisting}
Får fejl fordi du prøver at lægge en int med en float. Fsharp er meget pirlig omkring typer, så enten skal du konvertere 3 til 3.0 eller 1.0 til 1.

\section{ø.6}
På et stykke papir somewhere.

\section{ø.7}
Ditto.

\section{ø.8}
Ditto og det med tabeller kommer ikke til at ske' forløbigt. 

\section{ø.9}
\begin{lstlisting}
164uy + 230uy;;
> val it : byte = 138uy
\end{lstlisting}
Det der er sket her er underflow...\\
uy skal forstås som sbyte, en \textit{signed byte}, hvor byte bare er en byte.

\section{ø.10}
\begin{lstlisting}
"hello\n world\n"
> val it : string = "hello
 world
"
\end{lstlisting}
\escape{n} bruges til \textit{newline}. Men vidste ikke at det er en escape tingeling. 

\section{ø.11}
Eh?

\section{ø.12}


\section{ø.13}

\end{document}

